% 若编译失败,且生成 .synctex(busy) 辅助文件,可能有两个原因:
% 1. 需要插入的图片不存在:Ctrl + F 搜索 'figure' 将这些代码注释/删除掉即可
% 2. 路径/文件名含中文或空格:更改路径/文件名即可

% --------------------- 文章宏包及相关设置 --------------------- %
% >> ------------------ 文章宏包及相关设置 ------------------ << %
% 设定文章类型与编码格式
\documentclass[UTF8]{article}		

% 物理实验报告所需的其它宏包
\usepackage{ulem}   % \uline 下划线支持
\usepackage{circuitikz} % 电路图 tikz 支持

% 本 .tex 专属的宏定义
    \def\V{\ \mathrm{V}}
    \def\mV{\ \mathrm{mV}}
    \def\kV{\ \mathrm{KV}}
    \def\KV{\ \mathrm{KV}}
    \def\MV{\ \mathrm{MV}}
    \def\A{\ \mathrm{A}}
    \def\mA{\ \mathrm{mA}}
    \def\kA{\ \mathrm{KA}}
    \def\KA{\ \mathrm{KA}}
    \def\MA{\ \mathrm{MA}}
    \def\O{\ \Omega}
    \def\mO{\ \Omega}
    \def\kO{\ \mathrm{K}\Omega}
    \def\KO{\ \mathrm{K}\Omega}
    \def\MO{\ \mathrm{M}\Omega}
    \def\Hz{\ \mathrm{Hz}}

% 自定义宏定义
    \def\N{\mathbb{N}}
    \def\F{\mathbb{F}}
    \def\Z{\mathbb{Z}}
    \def\Q{\mathbb{Q}}
    \def\R{\mathbb{R}}
    \def\C{\mathbb{C}}
    \def\T{\mathbb{T}}
    \def\S{\mathbb{S}}
    %\def\A{\mathbb{A}}
    \def\I{\mathscr{I}}
    \def\d{\mathrm{d}}
    \def\p{\partial}


% 导入基本宏包
    \usepackage[UTF8]{ctex}     % 设置文档为中文语言
    \usepackage{hyperref}  % 宏包:自动生成超链接 (此宏包与标题中的数学环境冲突)
    \hypersetup{
        colorlinks=true,    % false:边框链接 ; true:彩色链接
        citecolor={blue},    % 文献引用颜色
        linkcolor={blue},   % 目录 (我们在目录处单独设置),公式,图表,脚注等内部链接颜色
        urlcolor={orange},    % 网页 URL 链接颜色,包括 \href 中的 text
        % cyan 浅蓝色 
        % magenta 洋红色
        % yellow 黄色
        % black 黑色
        % white 白色
        % red 红色
        % green 绿色
        % blue 蓝色
        % gray 灰色
        % darkgray 深灰色
        % lightgray 浅灰色
        % brown 棕色
        % lime 石灰色
        % olive 橄榄色
        % orange 橙色
        % pink 粉红色
        % purple 紫色
        % teal 蓝绿色
        % violet 紫罗兰色
    }
    % \usepackage{docmute}    % 宏包:子文件导入时自动去除导言区,用于主/子文件的写作方式,\include{./51单片机笔记}即可。注:启用此宏包会导致.tex文件capacity受限。
    \usepackage{amsmath}    % 宏包:数学公式
    \usepackage{mathrsfs}   % 宏包:提供更多数学符号
    \usepackage{amssymb}    % 宏包:提供更多数学符号
    \usepackage{pifont}     % 宏包:提供了特殊符号和字体
    \usepackage{extarrows}  % 宏包:更多箭头符号 
    \usepackage{multicol}   % 宏包:支持多栏 
    \usepackage{amsfonts}   % 宏包:数学字体

% 文章页面margin设置
    \usepackage[a4paper]{geometry}
        \geometry{top=0.75in}
        \geometry{bottom=0.75in}
        \geometry{left=0.75in}
        \geometry{right=0.75in}   % 设置上下左右页边距
        \geometry{marginparwidth=1.75cm}    % 设置边注距离(注释、标记等)

% 配置数学环境
    \usepackage{amsthm} % 宏包:数学环境配置
    % theorem-line 环境自定义
        \newtheoremstyle{MyLineTheoremStyle}% <name>
            {11pt}% <space above>
            {11pt}% <space below>
            {\kaishu}% <body font> 默认使用正文字体, \kaishu 为楷体
            {}% <indent amount>
            {\bfseries}% <theorem head font> 设置标题项为加粗
            {:\ \ }% <punctuation after theorem head>
            {.5em}% <space after theorem head>
            {\textbf{#1}\thmnumber{#2}\ \ (\,\textbf{#3}\,)}% 设置标题内容顺序
        \theoremstyle{MyLineTheoremStyle} % 应用自定义的定理样式
        \newtheorem{LineTheorem}{Theorem.\,}
    % theorem-block 环境自定义
        \newtheoremstyle{MyBlockTheoremStyle}% <name>
            {11pt}% <space above>
            {11pt}% <space below>
            {\kaishu}% <body font> 使用默认正文字体
            {}% <indent amount>
            {\bfseries}% <theorem head font> 设置标题项为加粗
            {:\\ \indent}% <punctuation after theorem head>
            {.5em}% <space after theorem head>
            {\textbf{#1}\thmnumber{#2}\ \ (\,\textbf{#3}\,)}% 设置标题内容顺序
        \theoremstyle{MyBlockTheoremStyle} % 应用自定义的定理样式
        \newtheorem{BlockTheorem}[LineTheorem]{Theorem.\,} % 使用 LineTheorem 的计数器
    % definition 环境自定义
        \newtheoremstyle{MySubsubsectionStyle}% <name>
            {11pt}% <space above>
            {11pt}% <space below>
            {}% <body font> 使用默认正文字体
            {}% <indent amount>
            {\bfseries}% <theorem head font> 设置标题项为加粗
            {:\\ \indent}% <punctuation after theorem head>
            {0pt}% <space after theorem head>
            {\textbf{#3}}% 设置标题内容顺序
        \theoremstyle{MySubsubsectionStyle} % 应用自定义的定理样式
        \newtheorem{definition}{}

%宏包:有色文本框(proof环境)及其设置
    \usepackage{xcolor}    %设置插入的文本框颜色
    \usepackage[strict]{changepage}     % 提供一个 adjustwidth 环境
    \usepackage{framed}     % 实现方框效果
        \definecolor{graybox_color}{rgb}{0.95,0.95,0.96} % 文本框颜色。修改此行中的 rgb 数值即可改变方框纹颜色,具体颜色的rgb数值可以在网站https://colordrop.io/ 中获得。(截止目前的尝试还没有成功过,感觉单位不一样)(找到喜欢的颜色,点击下方的小眼睛,找到rgb值,复制修改即可)
        \newenvironment{graybox}{%
        \def\FrameCommand{%
        \hspace{1pt}%
        {\color{gray}\small \vrule width 2pt}%
        {\color{graybox_color}\vrule width 4pt}%
        \colorbox{graybox_color}%
        }%
        \MakeFramed{\advance\hsize-\width\FrameRestore}%
        \noindent\hspace{-4.55pt}% disable indenting first paragraph
        \begin{adjustwidth}{}{7pt}%
        \vspace{2pt}\vspace{2pt}%
        }
        {%
        \vspace{2pt}\end{adjustwidth}\endMakeFramed%
        }

% 外源代码插入设置
    % matlab 代码插入设置
    \usepackage{matlab-prettifier}
        \lstset{style=Matlab-editor}    % 继承 matlab 代码高亮 , 此行不能删去
    \usepackage[most]{tcolorbox} % 引入tcolorbox包 
    \usepackage{listings} % 引入listings包
        \tcbuselibrary{listings, skins, breakable}
        \newfontfamily\codefont{Consolas} % 定义需要的 codefont 字体
        \lstdefinestyle{MatlabStyle_inc}{   % 插入代码的样式
            language=Matlab,
            basicstyle=\small\ttfamily\codefont,    % ttfamily 确保等宽 
            breakatwhitespace=false,
            breaklines=true,
            captionpos=b,
            keepspaces=true,
            numbers=left,
            numbersep=15pt,
            showspaces=false,
            showstringspaces=false,
            showtabs=false,
            tabsize=2,
            xleftmargin=15pt,   % 左边距
            %frame=single, % single 为包围式单线框
            frame=shadowbox,    % shadowbox 为带阴影包围式单线框效果
            %escapeinside=``,   % 允许在代码块中使用 LaTeX 命令 (此行无用)
            %frameround=tttt,    % tttt 表示四个角都是圆角
            framextopmargin=0pt,    % 边框上边距
            framexbottommargin=0pt, % 边框下边距
            framexleftmargin=5pt,   % 边框左边距
            framexrightmargin=5pt,  % 边框右边距
            rulesepcolor=\color{red!20!green!20!blue!20}, % 阴影框颜色设置
            %backgroundcolor=\color{blue!10}, % 背景颜色
        }
        \lstdefinestyle{MatlabStyle_src}{   % 插入代码的样式
            language=Matlab,
            basicstyle=\small\ttfamily\codefont,    % ttfamily 确保等宽 
            breakatwhitespace=false,
            breaklines=true,
            captionpos=b,
            keepspaces=true,
            numbers=left,
            numbersep=15pt,
            showspaces=false,
            showstringspaces=false,
            showtabs=false,
            tabsize=2,
        }
        \newtcblisting{matlablisting}{
            %arc=2pt,        % 圆角半径
            % 调整代码在 listing 中的位置以和引入文件时的格式相同
            top=0pt,
            bottom=0pt,
            left=-5pt,
            right=-5pt,
            listing only,   % 此句不能删去
            listing style=MatlabStyle_src,
            breakable,
            colback=white,   % 选一个合适的颜色
            colframe=black!0,   % 感叹号后跟不透明度 (为 0 时完全透明)
        }
        \lstset{
            style=MatlabStyle_inc,
        }

% table 支持
    \usepackage{booktabs}   % 宏包:三线表
    \usepackage{tabularray} % 宏包:表格排版
    \usepackage{longtable}  % 宏包:长表格

% figure 设置
    \usepackage{graphicx}  % 支持 jpg, png, eps, pdf 图片 
    \usepackage{svg}       % 支持 svg 图片
        \svgsetup{
            % 指向 inkscape.exe 的路径
            inkscapeexe = C:/aa_MySame/inkscape/bin/inkscape.exe, 
            % 一定程度上修复导入后图片文字溢出几何图形的问题
            inkscapelatex = false                 
        }
    \usepackage{subcaption} % 用于子图和小图注  

% 图表进阶设置
    \usepackage{caption}    % 图注、表注
        \captionsetup[figure]{name=图}  
        \captionsetup[table]{name=表}
        \captionsetup{
            labelfont=bf, % 设置标签为粗体
            textfont=bf,  % 设置文本为粗体
            font=small  
        }
    \usepackage{float}     % 图表位置浮动设置 

% 圆圈序号自定义
    \newcommand*\circled[1]{\tikz[baseline=(char.base)]{\node[shape=circle,draw,inner sep=0.8pt, line width = 0.03em] (char) {\small \bfseries #1};}}   % TikZ solution

% 列表设置
    \usepackage{enumitem}   % 宏包:列表环境设置
        \setlist[enumerate]{
            label=(\arabic*) ,   % 设置序号样式为加粗的 (1) (2) (3)
            ref=\arabic*, % 如果需要引用列表项,这将决定引用格式(这里仍然使用数字)
            itemsep=0pt, parsep=0pt, topsep=0pt, partopsep=0pt, leftmargin=3.5em} 
        \setlist[itemize]{itemsep=0pt, parsep=0pt, topsep=0pt, partopsep=0pt, leftmargin=3.5em}
        \newlist{circledenum}{enumerate}{1} % 创建一个新的枚举环境  
        \setlist[circledenum,1]{  
            label=\protect\circled{\arabic*}, % 使用 \arabic* 来获取当前枚举计数器的值,并用 \circled 包装它  
            ref=\arabic*, % 如果需要引用列表项,这将决定引用格式(这里仍然使用数字)
            itemsep=0pt, parsep=0pt, topsep=0pt, partopsep=0pt, leftmargin=3.5em
        }  

% 其它设置
    % 脚注设置
        \renewcommand\thefootnote{\ding{\numexpr171+\value{footnote}}}
    % 参考文献引用设置
        \bibliographystyle{unsrt}   % 设置参考文献引用格式为unsrt
        \newcommand{\upcite}[1]{\textsuperscript{\cite{#1}}}     % 自定义上角标式引用
    % 文章序言设置
        \newcommand{\cnabstractname}{序言}
        \newenvironment{cnabstract}{%
            \par\Large
            \noindent\mbox{}\hfill{\bfseries \cnabstractname}\hfill\mbox{}\par
            \vskip 2.5ex
            }{\par\vskip 2.5ex}

% 文章默认字体设置
    \usepackage{fontspec}   % 宏包:字体设置
        \setmainfont{SimSun}    % 设置中文字体为宋体字体
        \setCJKmainfont[AutoFakeBold=3]{SimSun} % 设置加粗字体为 SimSun 族,AutoFakeBold 可以调整字体粗细
        \setmainfont{Times New Roman} % 设置英文字体为Times New Roman

% 各级标题自定义设置
    \usepackage{titlesec}   
        % chapter 标题自定义设置
        \titleformat{\chapter}[hang]{\normalfont\huge\bfseries\centering\boldmath}{第\,\thechapter\,章}{20pt}{}
        \titlespacing*{\chapter}{0pt}{-20pt}{20pt} % 控制上下间距
        % section标题自定义设置 
        \titleformat{\section}[hang]{\normalfont\Large\bfseries\boldmath}{§\,\thesection\,}{8pt}{}
        % subsubsection标题自定义设置
        \titlespacing*{\subsubsection}{0pt}{3pt}{0pt} % 控制上下间距


% --------------------- 文章宏包及相关设置 --------------------- %
% >> ------------------ 文章宏包及相关设置 ------------------ << %


% ------------------------ 文章信息区 ------------------------ %
% ------------------------ 文章信息区 ------------------------ %
% 页眉页脚设置
\usepackage{fancyhdr}   %宏包:页眉页脚设置
    \pagestyle{fancy}
    \fancyhf{}
    \cfoot{\thepage}
    \renewcommand\headrulewidth{1pt}
    \renewcommand\footrulewidth{0pt}
    %\rhead{\bfseries \large {\color{red} 分组序号: 3-07}}    
    \chead{概率论与数理统计作业,\ 尹超,\ 2023K8009926003}
    \lhead{2024.12.8}

% 文档信息设置
%\title{这里是标题\\The Title of the Report}
%\author{尹超\\ \footnotesize 中国科学院大学,北京 100049\\ CarterYin \\ %\footnotesize University of Chinese Academy of Sciences, Beijing %100049, China}
%\date{\footnotesize 2024.8 -- 2025.1}

% 开始编辑文章

\begin{document}
%\noindent\begin{flushright}
%    \zihao{2}{分组序号: YK02-2}
%\end{flushright}

%\setCJKfamilyfont{boldsong}[AutoFakeBold = {2.17}]{SimSun}
%\newcommand*{\boldsong}{\CJKfamily{boldsong}}

\begin{center}\large
    \noindent{\huge\bfseries\bfseries《\ \ 概\ \ 率\ \ 论\ \ 与\ \ 数\ \ 理\ \ 统\ \ 计\ \ 作\ \ 业\ \ 》 }
    \\\vspace{0.4cm}
    %\noindent\textit{
    %    \textbf{\bfseries 实验名称:}\uline{\hspace{0.8cm} \bfseries 气轨上弹簧振子的简谐振动及瞬时速度的测定 \hspace{0.8cm}}\hspace{0.4cm} 
    %    指导教师:\uline{\hspace{0.5cm}某某\hspace{0.5cm}}}
    %\\\vspace{0.1cm}
    \noindent\textit{
        姓名:\uline{\,\,\,尹超\,\,\,}\hspace{0.2cm}
        学号:\uline{\,\,\,{\upshape 2023K8009926003}\,\,\,}\hspace{0.2cm}
        专业:\uline{\,\,\,人工智能\,\,\,}\hspace{0.2cm}
        班级:\uline{\,\,\,\upshape{2313}\,\,\,}
        %\,座号:\uline{\,\,\,\upshape{4}\,\,\,}
    }
    \\\vspace{0.1cm}
    %\noindent\textit{
    %    实验日期:\uline{\,\,{\upshape 2024.12.18}\,\,}\hspace{0.2cm}
    %    实验地点:\uline{\,\,\,教学楼{\upshape 716}\,\,\,}\hspace{0.2cm}
    %    是否调课/补课:\uline{\hspace{0.26cm}否 \hspace{0.26cm}}\hspace{0.2cm}
    %    成绩:\uline{\hspace{2cm}}}
\end{center}
% \vspace{-0.2cm}
\noindent\rule{\textwidth}{0.1em}   % 分割线
% ------------------------ 文章信息区 ------------------------ %
% ------------------------ 文章信息区 ------------------------ %


% 目录
\setcounter{tocdepth}{2}  % 目录深度为 2(不显示 subsubsection)
\noindent\tableofcontents\thispagestyle{fancy}   % 显示页码、页眉等

% 控制目录不换页
%\vspace{1cm}
%\setcounter{tocdepth}{2}  % 目录深度为 2(不显示 subsubsection)
%\noindent\begin{minipage}{\textwidth}
%\tableofcontents\thispagestyle{fancy}   % 显示页码、页眉等   
%\end{minipage}  

\newpage
%\rhead{\bfseries 分组序号: 3-07-4}



% >> --------------------- 下面是正文内容 --------------------- << %
% ------------------------ 下面是正文内容 ------------------------ %
% ------------------------ 下面是正文内容 ------------------------ %
% ------------------------ 下面是正文内容 ------------------------ %
% ------------------------ 下面是正文内容 ------------------------ %
% >> --------------------- 下面是正文内容 --------------------- << %

\section{12.02}
\subsection*{题目描述}
\textbf{Show that in simple linear regression,}
\[
\text{Coefficient of Determination} = (\text{Coefficient of correlation})^2
\]
\textbf{Where Coefficient of Determination is defined as:}
\[
\text{Coefficient of Determination} = \frac{SS_{\text{total}} - SS_{\text{err}}}{SS_{\text{total}}}
\]

\textbf{证明在简单线性回归中,}
\[
\text{决定系数} = (\text{相关系数})^2
\]
\textbf{其中决定系数定义为:}
\[
\text{决定系数} = \frac{SS_{\text{total}} - SS_{\text{err}}}{SS_{\text{total}}}
\]

\subsection*{解答过程}

在简单线性回归中,我们将因变量 \( y \) 与自变量 \( x \) 之间的关系建模为:

\[
y = \beta_0 + \beta_1 x + \epsilon
\]

其中:
- \( \beta_0 \) 是截距。
- \( \beta_1 \) 是斜率。
- \( \epsilon \) 是误差项。

决定系数(R²)定义为:

\[
R^2 = \frac{SS_{\text{total}} - SS_{\text{err}}}{SS_{\text{total}}}
\]

其中:
- \( SS_{\text{total}} \) 是 \( y \) 的总平方和,表示数据的总方差。
- \( SS_{\text{err}} \) 是残差平方和,表示回归模型无法解释的方差。

相关系数(r)是皮尔逊相关系数,用来衡量 \( x \) 和 \( y \) 之间线性关系的强度。在简单线性回归中,相关系数 \( r \) 的公式为:

\[
r = \frac{\text{Cov}(x, y)}{\sigma_x \sigma_y}
\]

其中:
- \( \text{Cov}(x, y) \) 是 \( x \) 和 \( y \) 的协方差。
- \( \sigma_x \) 和 \( \sigma_y \) 分别是 \( x \) 和 \( y \) 的标准差。

决定系数(R²)可以写成:

\[
R^2 = \frac{SS_{\text{reg}}}{SS_{\text{total}}}
\]

其中回归平方和 \( SS_{\text{reg}} \) 是总平方和和误差平方和的差:

\[
SS_{\text{reg}} = SS_{\text{total}} - SS_{\text{err}} = \sum_{i=1}^{n} (\hat{y}_i - \bar{y})^2
\]

因此,决定系数 \( R^2 \) 与相关系数 \( r \) 的关系是:

\[
r^2 = \frac{SS_{\text{reg}}}{SS_{\text{total}}}
\]

所以我们可以得出结论:

\[
R^2 = r^2
\]

这证明了在简单线性回归中,决定系数(R²)等于相关系数(r)的平方。\\

\cleardoublepage

\subsection*{题目描述}
假设 $Y_i = b_0 + b_1 x_i + e_i$,其中 $e_1, e_2, \ldots, e_n$ 独立同分布,且 $e_i \sim N(0, \sigma^2)$。

找到 $b_0$ 和 $b_1$ 的最大似然估计并验证它们是最小二乘估计。

\subsection*{解答过程}

首先,写出似然函数:
\[
L(b_0, b_1, \sigma^2) = \prod_{i=1}^n \frac{1}{\sqrt{2\pi\sigma^2}} \exp\left(-\frac{(Y_i - b_0 - b_1 x_i)^2}{2\sigma^2}\right)
\]

取对数得到对数似然函数:
\[
\ell(b_0, b_1, \sigma^2) = -\frac{n}{2} \log(2\pi\sigma^2) - \frac{1}{2\sigma^2} \sum_{i=1}^n (Y_i - b_0 - b_1 x_i)^2
\]

对 $b_0$ 和 $b_1$ 求偏导数并令其等于零:
\[
\frac{\partial \ell}{\partial b_0} = \frac{1}{\sigma^2} \sum_{i=1}^n (Y_i - b_0 - b_1 x_i) = 0
\]
\[
\frac{\partial \ell}{\partial b_1} = \frac{1}{\sigma^2} \sum_{i=1}^n (Y_i - b_0 - b_1 x_i)x_i = 0
\]

解这两个方程可以得到 $b_0$ 和 $b_1$ 的最大似然估计:
\[
b_1 = \frac{\sum_{i=1}^n (x_i - \bar{x})(Y_i - \bar{Y})}{\sum_{i=1}^n (x_i - \bar{x})^2}
\]
\[
b_0 = \bar{Y} - b_1 \bar{x}
\]

这正是最小二乘估计的公式,因此最大似然估计和最小二乘估计是一致的。






\cleardoublepage

\section{12.04}

\subsection*{题目描述}

在简单线性回归模型中,证明 ANOVA 中的 F 统计量是用于检验 $H_0: b_1 = 0$ 与 $H_1: b_1 \neq 0$ 的 T 统计量的平方。

\subsection*{证明}

在简单线性回归中,模型为:
\[
Y_i = b_0 + b_1 x_i + e_i
\]

假设检验 $H_0: b_1 = 0$ 与 $H_1: b_1 \neq 0$,T 统计量定义为:
\[
t = \frac{b_1}{\text{SE}(b_1)}
\]

其中,$\text{SE}(b_1)$ 是 $b_1$ 的标准误。

ANOVA 中的 F 统计量定义为:
\[
F = \frac{\text{MS}_{\text{reg}}}{\text{MS}_{\text{err}}}
\]

其中,$\text{MS}_{\text{reg}}$ 是回归均方,$\text{MS}_{\text{err}}$ 是误差均方。

在简单线性回归中,$\text{MS}_{\text{reg}}$ 和 $\text{MS}_{\text{err}}$ 可以表示为:
\[
\text{MS}_{\text{reg}} = \frac{SS_{\text{reg}}}{1}
\]
\[
\text{MS}_{\text{err}} = \frac{SS_{\text{err}}}{n-2}
\]

其中,$SS_{\text{reg}}$ 是回归平方和,$SS_{\text{err}}$ 是误差平方和。

由于 $SS_{\text{reg}} = b_1^2 \sum_{i=1}^n (x_i - \bar{x})^2$,我们有:
\[
F = \frac{b_1^2 \sum_{i=1}^n (x_i - \bar{x})^2}{\frac{SS_{\text{err}}}{n-2}}
\]

注意到 $\text{SE}(b_1) = \sqrt{\frac{SS_{\text{err}}}{(n-2) \sum_{i=1}^n (x_i - \bar{x})^2}}$,我们可以得到:
\[
t^2 = \left(\frac{b_1}{\text{SE}(b_1)}\right)^2 = \frac{b_1^2}{\frac{SS_{\text{err}}}{(n-2) \sum_{i=1}^n (x_i - \bar{x})^2}} = F
\]

因此,ANOVA 中的 F 统计量是用于检验 $H_0: b_1 = 0$ 与 $H_1: b_1 \neq 0$ 的 T 统计量的平方。

















\end{document}































% VScode 常用快捷键:

% F2:                       变量重命名
% Ctrl + Enter:             行中换行
% Alt + up/down:            上下移行
% 鼠标中键 + 移动:           快速多光标
% Shift + Alt + up/down:    上下复制
% Ctrl + left/right:        左右跳单词
% Ctrl + Backspace/Delete:  左右删单词    
% Shift + Delete:           删除此行
% Ctrl + J:                 打开 VScode 下栏(输出栏)
% Ctrl + B:                 打开 VScode 左栏(目录栏)
% Ctrl + `:                 打开 VScode 终端栏
% Ctrl + 0:                 定位文件
% Ctrl + Tab:               切换已打开的文件(切标签)
% Ctrl + Shift + P:         打开全局命令(设置)

% Latex 常用快捷键:

% Ctrl + Alt + J:           由代码定位到PDF